\section{Related Work}

Acoustic source localization in sensor network is a widely-studied problem. 
In the past few years, there has been a growing interest for spatial distributions of independent (unsynchronized) acoustic sensors, each made of two or more synchronized microphones. Due to space constraints, we can only mention a few directly related works here.
% Omologo, \emph{et al.} \cite{omologo1994acoustic} compute the Steered Response Power maps associated to all the microphone pairs over a spatial grid and then localize the source
% as the peak of the cumulative global map, with overall computational costs that are often too demanding for the application at hand.
% Better computational efficiency is achieved in \cite{cobos2011modified} where the SRP  accommodates a different computation over a coarser grid.
% Alternate approaches based on Least Squares were proposed in \cite{rabinkin1996dsp} with a certain sensitivity to environmental noise;
% and in \cite{do2007real} where a stochastic region contraction of the grid was proposed, adopting a multi-resolution approach.
Wang, \emph{et al.} \cite{wang2003acoustic} described a system having static cluster architecture, the system experienced a problem in that the accuracy decreased when an acoustic source occurred between the clusters.
Chen, \emph{et al.} \cite{chen2004dynamic} showed that nodes in the system did not need to recognize their cluster head, reducing the constraints on deployment of the localization system.
Hu, \emph{et al.} \cite{hu2009design} design the system based on 2-tier architecture, which experienced cost and deployment problems especially in the very large target area.
Rabbat, \emph{et al.} \cite{rabbat2005robust} proposed a decentralized algorithm based on the distributed ML estimation technique using token ring architecture.
Kim, \emph{et al.} \cite{kim2009locating}proposed to identify the node closest to the acoustic source, based on TOA comparisons between all nodes, thus incurring high communication cost and requiring global synchronization between all sensor nodes.
Lightning is a method proposed in \cite{wang2008lightning} to identify the sensor closest to the acoustic source, also based on expensive broadcasting/flooding.
Aarabi, \emph{et al.} ~\cite{aarabi1900fusion} used 10 dual-microphone arrays distributed in a room and used their data to locate three speakers.
Wu, \emph{et al.} ~\cite{wu2012fusion} used three dual-microphone arrays to locate two sound sources in a distributed way in which only the local DOA estimates are communicated among arrays.
Canclini, \emph{et al.}\cite{canclini2013acoustic,Canclini2015} proposed a method for localizing an acoustic source with distributed microphone networks based on TDOA between microphones of the same sensor.

Most of the existing acoustic source localization methods in sensor networks are based on range-based measurement.
In contrast, our work is a range-free method and shown to be robust to the errors of node locations and the errors of measurements.
There have existed some research on range-free localization method.
Yedavalli, \emph{et al.} ~\cite{yedavalli2008sequence} proposed a Sequence-Based Localization(SBL) in WSN. 
The heart of SBL is the division of a 2D localization space into distinct regions by the perpendicular bisectors of lines joining pairs of reference nodes (nodes with known locations).
%Each distinct region formed in this manner can be uniquely identified by a location sequence that represents the distance ranks of reference nodes to that region. 
%The unknown node first determines its own location sequence based on the measurement between itself and the reference nodes, then searches through the location sequence table to determine its location.
In their earlier work \cite{yedavalli2005ecolocation}, Ecolocation used location constraints for robust localization.
%A location constraint is a relationship between the distances of two reference nodes from the unknown node that determines its proximity to either
%reference nodes. Location constraints can be graphically represented by perpendicular bisectors
%between reference nodes, and each location sequence can be written as a set of location constraints.
%Thus, the location constraint set is also unique to each region in the arrangement.
Chakrabarty, \emph{et al.} \cite{chakrabarty2002grid} and Ray, \emph{et al.} \cite{ray2004robust} use identity codes to determine the location of sensor nodes in grid and nongrid sensor fields, respectively. 
% Here, each grid point or region in the localization space is identified by a unique set of reference node IDs whose signals can reach
% the point or region, and this unique set is an identity code for that point or region. 
% The two main drawbacks of this
% approach are that 1) in order to uniquely identify all unknown node locations in the localization space, the reference nodes need to be placed carefully according to
% rules determined by an optimization algorithm, and that 2) for acceptable location accuracies, the number of reference nodes required is prohibitively expensive, and for
% sparse networks of reference nodes, the accuracy is coarse grained in the order of radio range. 
He, \emph{et al.} \cite{he2003range} propose an RF-based localization technique in which the unknown node location is determined by the intersection of all triangles,
formed by reference nodes, that are likely to bound it. The unknown node determines its existence inside a triangle by
comparing its measured RSS values to that of its neighbors to detect a trend in RSS values in any particular direction
This technique depends on the weak assumption that signal strength decreases monotonically with distance, which is not true in real-world scenarios.
Zhong, \emph{et al.} \cite{zhong2009tracking} convert the original tracking problem to the problem of finding the shortest path in a graph, which is equivalent to optimal matching of a series of node sequences. 
Zhong, \emph{et al.} \cite{zhong2009achieving} introduce a proximity metric called RSD to capture the distance relationships
among 1-hop neighboring nodes in a range-free manner. With little overhead, RSD can be conveniently applied as a transparent
supporting layer for state-of-the-art connectivity-based localization solutions to achieve better accuracy.




